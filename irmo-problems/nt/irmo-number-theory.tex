\documentclass{pset}

\usepackage{xcolor}
\usepackage{hyperref}

\definecolor{hintpink}{RGB}{219, 48, 122}


\course{IrMO Number Theory}
\name{Adam Kelly}
\due{\today}

% \blurb{\textbf{Remark.} This is a collection of all functional equation problems that have appeared in the Irish Mathematical Olympiad.
% The questions are ordered chronologically. All problems are due to their respective creators.}

 \blurb{\textbf{Remark.} This is a collection of all number theory problems that have appeared in the Irish Mathematical Olympiad and the Irish EGMO selection test.
 The questions are ordered chronologically. All problems are due to their respective creators.}


\newenvironment{hint}% environment name
{% begin code
\color{hintpink}
\noindent
  \begin{itshape}\textbf{Hint:}%
}%
{\end{itshape}}% end code

\begin{document}

\section*{EGMO Selection Test Problems}

\begin{problems}
    \begin{problem}[EGMO TST 2020]
    Let \(a, b, c\) be integers such that \(a-c\) is even and \(b-c\) is divisible by \(3 .\) Show that
    $$
        \frac{a n^{2}}{2}+\frac{b n^{3}}{3}+\frac{c n}{6}
    $$
    is an integer for every integer \(n\).
    \end{problem}

    \begin{problem}[EGMO TST 2020]
    Note that the integers \(6,10,15\) have the property that any two of them have a common divisor greater than
    \(1,\) but the only common divisor of all three is 1
    \begin{enumerate}
        \item Find four integers with the property that any pair of them has a common divisor greater than \(1,\) but no triple of them has a common divisor greater than \(1 .\)
        \item Do there exist 2020 integers with the property that every collection of 1010 of them has a common divisor greater than \(1,\) but no collection of 1011 of them has a common divisor greater than \(1 ?\)
    \end{enumerate}
    \end{problem}

    \begin{problem}[EGMO TST 2020]
    The triple \((1,5,7)\) is such that the squares \((1,25,49)\) are in arithmetic progression. Show that there are infinitely many triples of positive integers \((a, b, c)\) with greatest common divisor 1 such that \(a^{2}, b^{2}\) and \(c^{2}\)
    are in arithmetic progression.
    \end{problem}

    \begin{problem}[EGMO TST 2020]
    Show that there are no integers \(x, y\) satisfying \(x^{2}+x y-3 y^{2}=2020\)
    \end{problem}

    \begin{problem}[EGMO TST 2019]
    Finn has 5 distinct real numbers. He takes the sum of each pair of numbers and writes down the 10 sums. The 3 smallest sums are \(30,34\) and \(35,\) while the 2 largest are 46 and \(49 .\)

    Determine, with proof, the largest of Finn's 5 numbers.
    \end{problem}

    \begin{problem}[EGMO TST 2019]
    For an integer \(r \geq 2,\) define \(s(r)\) to be the smallest prime number that divides \(r\) Show that for any integer \(n \geq 2\)
    $$
        \sum_{r=2}^{n} s(r) \geq 3 n-5
    $$
    \end{problem}

    \begin{problem}[EGMO TST 2018]
    Are there any positive integers \(n\) and \(m\) such that the integer \(32^{n}+3125^{m}\) is a prime number?
    \end{problem}

    \begin{problem}[EGMO TST 2018]
    How many different pairs of integers \((x, y)\) satisfy the equation
    $$
        10 x^{2}+29 x y+21 y^{2}=15 ?
    $$
    Write down 3 such pairs.
    \end{problem}

    \begin{problem}[EGMO TST 2018]
    Suppose \(n\) is a positive integer, such that all the digits of \(72 n,\) written in decimal notation, are \(0^{\prime} \mathrm{s}\) and 1 's. Find the smallest such \(n .\)
    \end{problem}

    \begin{problem}[EGMO TST 2017]
    A positive integer is said to be near-square if it is a product of two positive integers differing by \(1 .\) For example, 20 is a near-square because \(20=4 \times 5 .\) Prove that every near-square integer can be expressed as the ratio of two other near-square positive integers.
    \end{problem}

    \begin{problem}[EGMO TST 2017]
    Determine with proof all prime numbers \(p\) for which \(7 p+4\) is the square of an integer.
    \end{problem}

    \begin{problem}[EGMO TST 2017]
    \begin{enumerate}
        \item Simplify \(\left(x^{2}-1\right)^{2}+\left(x^{2}+2 x\right)^{2}-\left(x^{2}+x+1\right)^{2}\) and then factor the result as far as possible.
        \item Show that there are infinitely many pairs of integers \(m, n\) for which \(m^{2}+n^{2}-m n\) is the square of an integer.

    \end{enumerate}
    \end{problem}



    \begin{problem}[EGMO TST 2017]
    Let \(p, q, r\) be prime numbers with
    $$
        p<q<r<q+p^{4} \quad \text { and } \quad p q^{2}=r^{2}+1
    $$
    Find, with proof, all possible values for \(p, q\) and \(r\)
    \end{problem}

    \begin{problem}[EGMO TST 2016]
    The function \(\mu\) is defined on the set of positive integers as follows:
    \begin{itemize}
        \item \(\mu(1)=1\) and \(\mu(p)=-1\) for any prime number \(p\)
        \item \(\mu(a b)=\mu(a) \mu(b)\) for any positive integers with \(\operatorname{gcd}(a, b)=1\)
        \item \(\mu(n)=0\) if \(n\) is a positive integer which is divisible with a square of a prime number.
    \end{itemize}
    (For instance \(\mu(15)=\mu(3) \mu(5)=1\) and \(\mu(12)=0,\) because 12 is divisible with \(2^{2}\) ).
    Prove that for any positive integer \(n>1,\) we have \(\sum_{d | n} \mu(d)=0\)
    \end{problem}

    \begin{problem}[EGMO TST 2016]
    I have two egg timers. The first can time an interval of exactly 7 minutes. The second can time an interval of exactly 9 minutes. Explain how I can use them to boil an egg for exactly 3 minutes?
    \end{problem}

    \begin{problem}[EGMO TST 2016]
    \begin{enumerate}
        \item Show that the greatest common divisor of \((n+1) !+1\) and \(n !+1\) is \(1,\) for all integers \(n \geq 1\)
        \item For any \(n>1,\) find integers \(x, y\) such that
              $$
                  ((n+1) !+1) x+(n !+1) y=1
              $$
              [Recall that \(n !=1 \times 2 \times \cdots \times n \text { for any } n>1 .]\)
    \end{enumerate}
    \end{problem}

    \begin{problem}[EGMO TST 2016]
    Richard and nine other people are standing in a circle. All ten of them think of an integer (that may be negative) and whisper their number to both of their neighbours. Afterwards, they each state the average of the two numbers that were whispered in their ear.

    Richard states the number \(10,\) his right neighbour states the number \(9,\) the next person along the circle states the number \(8,\) and so on, finishing with Richard's left neighbour who states the number \(1 .\)

    What number did Richard have in mind?
    \end{problem}

    \begin{problem}[EGMO TST 2016]
    For each positive integer \(n\) let \(s_{n}=n !+20 !\)
    \begin{enumerate}
        \item Let \(q>20\) be a prime number. Prove that there are only a finite number of positive integers \(k\) for which \(q\) divides \(s_{k}\)
        \item Find with proof all prime numbers \(p\) for which there exists a positive integer \(m\) such that \(p\) divides \(s_{m}\) and \(s_{m + 1}\).
    \end{enumerate}
    \end{problem}

    \begin{problem}[EGMO TST 2016]
    For a real number \(x\) denote by \([x]\) the greatest integer not exceeding \(x\)
    \begin{enumerate}
        \item Find with proof all positive integers \(k\) for which \([\sqrt[3]{k^{3}+20 k}] \neq k\)
        \item Prove that if \(n\) is a positive integer, then \(\left[n+\sqrt{n}+\frac{1}{2}\right]\) is not the square of an integer.
    \end{enumerate}
    \end{problem}

    \begin{problem}[EGMO TST 2015]
    Determine all triples \((a, b, c)\) of positive integers satisfying both of the following properties:
    \begin{itemize}
        \item We have \(a<b<c,\) and \(a, b\) and \(c\) are three consecutive odd integers;
        \item The number \(a^{2}+b^{2}+c^{2}\) consists of four equal digits.
    \end{itemize}
    \end{problem}

    \begin{problem}[EGMO TST 2015]
    \begin{enumerate}

        \item Find with proof all integers \(x, y\) such that
              $$
                  \frac{x^{2}+x^{2}+x^{2}}{3}
              $$
              is a prime number.
        \item Prove that if \(x\) and \(y\) are integers, then
              $$
                  \frac{x^{4}+x^{2} y^{2}+y^{4}}{5}
              $$
              is not a prime number.
    \end{enumerate}
    \end{problem}

    \begin{problem}[EGMO TST 2015]
    A triangle has angles of \(36^{\circ}, 72^{\circ}\) and \(72^{\circ} .\) Prove that it has at least one side whose length is not
    an integer.
    \end{problem}

    \begin{problem}[EGMO TST 2015]
    Find with proof all positive integers \(k\) such that, for \(n=2^{k},\) every prime number which divides \(n !+1\) also divides \(n+1\)
    \end{problem}

    \begin{problem}[EGMO TST 2014]
    \begin{enumerate}

        \item  Prove that if \(n\) is a positive integer, then \(n^{n}+(n+1)^{n}\) is not divisible by 2014
        \item  Find a positive integer \(m\) for which \(m^{m}+(m+2)^{m}\) is divisible by 2014
        \item  Find with proof the least positive integer \(k\) for which \(k^{k}(k+1)^{k}\) is divisible by 2014

    \end{enumerate}
    \end{problem}


    \begin{problem}[EGMO TST 2014]
    The length of each side of a triangle is an integer and is a divisor of the perimeter of the triangle. Prove that the triangle is equilateral.
    \end{problem}

    \begin{problem}[EGMO TST 2014]
    Show that it is not possible to find 14 consecutive integers such that each of them is divisible by at least one of the numbers \(2,3,5,7,11\)
    \end{problem}

    \begin{problem}[EGMO TST 2014]
    Show that
    $$
        1+\frac{1}{2}+\cdots+\frac{1}{n}
    $$
    is not an integer for any \(n>1\)
    \end{problem}

    \begin{problem}[EGMO TST 2014]
    Let \(\left\{p_{n}\right\}\) be the increasing sequence of prime numbers, that is,
    \(p_{1}=2, p_{2}=3, p_{3}=5, p_{4}=7, \ldots\)
    Prove that for all integers \(k>2,\) we have
    \begin{enumerate}
        \item \(p_{k+9} \geq 3 k+25\)
        \item \(p_{k+1} \leq 1+p_{1} p_{2} \cdots p_{k}\)
        \item \(p_{k+1}<p_{k-1} \sqrt{p_{1} p_{2} \cdots p_{k}}\)
    \end{enumerate}
    \end{problem}

    \begin{problem}[EGMO TST 2013]
    We say that a positive integer is triangular if it is the sum of some positive consecutive integers starting from 1 (thus, 1 \(=1,3=1+2,6=1+2+3,10=1+2+3+4\) are triangular numbers). Prove that if \(n\) is triangular, then so is \(25 n+3\)
    \end{problem}

    \begin{problem}[EGMO TST 2013]
    We are given a set \(X\) containing 100 integers, none of which is divisible by \(3 .\) We are asked to carry out the following task: choose 7 integers from this set so that for any pair of integers \(x\) and \(y\) we choose, the difference \(x-y\) is not divisible by \(9 .\)
    \begin{enumerate}
        \item Prove that this task is impossible.
        \item If we are instead asked to choose 6 integers from \(X,\) is the task always possible?
    \end{enumerate}
    \end{problem}

    \begin{problem}[EGMO TST 2013]
    Let \(x, y\) be positive integers with
    $$
        3 x+4 y+x y=2012
    $$
    \begin{enumerate}
        \item Prove that \(x+y \geq 83\)
        \item Prove also that the same inequality is valid if 2012 is replaced by 2013 .
    \end{enumerate}
    \end{problem}

    \begin{problem}[EGMO TST 2013]
    We would like to place stamps worth exactly \(n\) cents on an envelope. However, there are only 5 -cent and 12 -cent stamps available to us (although we have an unlimited supply of both of these stamps). Prove that we can perform the task provided \(n \geq 44\)
    \end{problem}

    \begin{problem}[EGMO TST 2012]
    Suppose 251 numbers are chosen from \(1,2,3, \ldots, 499,500 .\) Show that, no matter how the numbers are chosen, there must be two that are consecutive.
    \end{problem}

    \begin{problem}[EGMO TST 2012]
    Prove or disprove: For every positive integer \(n,\) the greatest common divisor of \(5 n+4\) and \(9 n-7\) is 1
    \end{problem}

    \begin{problem}[EGMO TST 2012]
    Suppose that \(a\) and \(m\) are integers larger than \(1 .\) Prove that the greatest common divisor of the pair \(a-1\) and \(m\) is equal to the greatest common divisor of the pair of integers \(a-1\) and \(\left(a^{m}-1\right) /(a-1)\)
    \end{problem}

    \begin{problem}[EGMO TST 2012]
    We say that a positive integer is triangular if it is the sum of some positive consecutive integers starting from \(1(\text { thus, } 1=1,3=1+2,6=1+2+3,10=1+2+3+4\) are triangular numbers). We denote by
    $$
        t_{1}<t_{2}<\cdots<t_{n}<\cdots
    $$
    the sequence of all triangular numbers. Prove that for all \(n \geq 1\) we have
    $$
        1 \cdot t_{1}+2 \cdot t_{2}+\ldots n \cdot t_{n}=\frac{n(n+1)(n+2)(3 n+1)}{24}
    $$
    \end{problem}

    \begin{problem}[EGMO TST 2012]
    Let \(k\) be a positive integer and \(p_{1}=2<p_{2}<\cdots<p_{k}\) be the first \(k\) prime numbers and
    let \(M=1+p_{1} p_{2} \cdots p_{k}\)
    Prove the following:
    \begin{enumerate}
        \item \(M\) is not the square of an integer;
        \item \(M\) is not the cube of an integer;
        \item \(M\) is not the \(q^{\text {th }}\) power of an integer for any \(q \in\left\{p_{1}, p_{2}, \ldots, p_{k}\right\}\)
    \end{enumerate}
    \end{problem}

    \begin{problem}[EGMO TST 2012]
    \begin{enumerate}

        \item Prove that there exist infinitely many primes of the form \(6 n-1,\) with \(n\) an integer.
        \item Let \(S\) be the set of all integers of the form \(a^{2}+a b+b^{2},\) where \(a\) and \(b\) are integers. Prove the following:
              \begin{enumerate}

                  \item If \(x\) and \(y\) are in \(S,\) then \(x y\) is in \(S\);
                  \item There exist infinitely many primes which are not in \(\mathrm{S}\).
              \end{enumerate}
    \end{enumerate}
    \end{problem}

\end{problems}

\clearpage

\section*{IrMO Problems}

\begin{problems}

\begin{problem}[IrMO 2020 Q8]
    Determine the last (rightmost) three decimal digits of $n$ where:
$$
n=1 \times 3 \times 5 \times 7 \times \ldots \times 2019.
$$
\end{problem}

    \begin{problem}[IrMO 2020 Q7]
        Let $\mathrm{N}$ denote the strictly positive integers. $\mathrm{A}$ function $f: \mathbb{N} \rightarrow \mathbb{N}$ satisfies the following for all $n \in \mathbb{N}$
\begin{align*}
f(1) &=1 \\
f(f(n)) &=n \\
f(2 n) &=2 f(n)+1.
\end{align*}
Find the value of $f(2020)$.
    \end{problem}

    \begin{problem}[IrMO 2020 Q1]
        We say an integer $n$ is \emph{naoish} if $n \geq 90$ and the second-to-last digit of $n$ (in decimal notation) is equal to $9 .$ For example, 10798, 1999 and 90 are naoish, whereas 9900, 2009 and 9 are not. Nino expresses 2020 as a sum:
$$
2020=n_{1}+n_{2}+\ldots+n_{k}
$$
where each of the $n_{j}$ is naoish.

What is the smallest positive number $k$ for which Nino can do this?
    \end{problem}

    \begin{problem}[IrMO 2019 Q1]
        Define the quasi-primes as follows.
        \begin{itemize}
            \item The first quasi-prime is \(q_{1}=2\)
\item For \(n \geq 2,\) the \(n^{\text {th }}\) quasi-prime \(q_{n}\) is the smallest integer greater than \(q_{n-1}\) and
not of the form \(q_{i} q_{j}\) for some \(1 \leq i \leq j \leq n-1\) 
        \end{itemize}

Determine, with proof, whether or not 1000 is a quasi-prime.
    \end{problem}

\begin{problem}[IrMO 2019 Q6]
    The number 2019 has the following nice properties:
    \begin{enumerate}
    \item It is the sum of the fourth powers of five distinct positive integers.
    \item It is the sum of six consecutive positive integers.
    \end{enumerate}
    In fact,
    \begin{align*}
    2019&=1^{4}+2^{4}+3^{4}+5^{4}+6^{4} \\
    2019&=334+335+336+337+338+339
    \end{align*}

    Prove that 2019 is the smallest number that satisfies both (a) and (b). (You may assume that ( 1) and ( 2) are correct!)
\end{problem}

\begin{problem}[IrMO 2018 Q1]
    Mary and Pat play the following number game. Mary picks an initial integer greater than \(2017 .\) She then multiplies this number by 2017 and adds 2 to the result. Pat will add 2019 to this new number and it will again be Mary's turn. Both players will continue to take alternating turns. Mary will always multiply the current number by 2017 and add 2 to the result when it is her turn. Pat will always add 2019 to the current number when it is his turn. Pat wins if one of the numbers obtained is divisible by \(2018 .\) Mary wants to prevent Pat from winning the game. Determine, with proof, the smallest initial integer Mary could choose in order to achieve this.
\end{problem}

\begin{problem}[IrMO 2018 Q9]
    The sequence of positive integers \(a_{1}, a_{2}, a_{3}, \ldots\) satisfies
    $$
    a_{n+1}=a_{n}^{2}+2018 \quad \text { for } n \geq 1
    $$
    Prove that there exists at most one \(n\) for which \(a_{n}\) is the cube of an integer.
\end{problem}

\begin{problem}[IrMO 2017 Q1]
    Determine, with proof, the smallest positive multiple of 99 all of whose digits are either 1 or 2
\end{problem}

\begin{problem}[IrMO 2017 Q5]
    The sequence \(a=\left(a_{0}, a_{1}, a_{2}, \ldots\right)\) is defined by \(a_{0}=0, a_{1}=2\) and
    $$
    a_{n+2}=2 a_{n+1}+41 a_{n} \text { for all } n \geq 0
    $$
    Prove that \(a_{2016}\) is divisible by 2017 .
\end{problem}

\begin{problem}[IrMO 2017 Q6]
    Does there exist an even positive integer \(n\) for which \(n+1\) is divisible by 5 and the two numbers \(2^{n}+n\) and \(2^{n}-1\) are co-prime?
\end{problem}

\begin{problem}[IrMO 2016 Q1]
    If the three-digit number \(A B C\) is divisible by \(27,\) prove that the three-digit numbers \(B C A\) and \(C A B\) are also divisible by 27
\end{problem}

\begin{problem}[IrMO 2016 Q7]
    A rectangular array of positive integers has four rows. The sum of the entries in each column is \(20 .\) Within each row, all entries are distinct. What is the maximum possible number of columns?
\end{problem}

\begin{problem}[IrMO 2016 Q9]
    Show that the number
    $$
    \left(\frac{251}{\frac{1}{\sqrt[3]{252}-5 \sqrt[3]{2}}-10 \sqrt[3]{63}}+\frac{1}{\frac{251}{\sqrt[3]{252}+5 \sqrt[3]{2}}+10 \sqrt[3]{63}}\right)^{3}
    $$
    is an integer and find its value.
\end{problem}

\begin{problem}[IrMO 2015 Q3]
    Find all positive integers \(n\) for which both \(837+n\) and \(837-n\) are cubes of positive integers.
\end{problem}

\begin{problem}[IrMO 2015 Q7]
    Let \(n>1\) be an integer and \(\Omega:=\{1,2, \ldots, 2 n-1,2 n\}\) the set of all positive integers that are not larger than \(2 n .\)

    A nonempty subset \(S\) of \(\Omega\) is called sum-free if, for all elements \(x, y\) belonging
    to \(S, x+y\) does not belong to \(S .\) We allow \(x=y\) in this condition. Prove that \(\Omega\) has more than \(2^{n}\) distinct sum-free subsets.
\end{problem}

\begin{problem}[IrMO 2014 Q2]
    Prove for all integers \(N>1\) that \(\left(N^{2}\right)^{2014}-\left(N^{11}\right)^{106}\) is divisible by \(N^{6}+N^{3}+1\)
\end{problem}

\begin{problem}[IrMO 2014 Q2]
    Each of the four positive integers \(N, N+1, N+2, N+3\) has exactly six positive divisors. There are exactly 20 different positive numbers which are exact divisors of at least one of the numbers. One of these is \(27 .\) Find all possible values of \(N\) (Both 1 and \(m\) are counted as divisors of the number \(m .\) )
\end{problem}

\begin{problem}[IrMO 2013 Q1]
    Find the smallest positive integer \(m\) such that \(5 m\) is an exact \(5^{\text {th }}\) power, \(6 m\) is an exact \(6^{\text {th }}\) power, and \(7 m\) is an exact \(7^{\text {th }}\) power.
\end{problem}

\begin{problem}[IrMO 2013 Q8]
    Find the smallest positive integer \(N\) for which the equation \(\left(x^{2}-1\right)\left(y^{2}-1\right)=N\) is satisfied by at least two pairs of integers \((x, y)\) with \(1<x \leq y .\)
\end{problem}

\begin{problem}[IrMO 2012 Q6]
    Let \(S(n)\) be the sum of the decimal digits of \(n .\) For example, \(S(2012)=2+0+1+2=\)
    5. Prove that there is no integer \(n>0\) for which \(n-S(n)=9990\)
\end{problem}

\begin{problem}[IrMO 2011 Q3]
    The integers \(a_{0}, a_{1}, a_{2}, a_{3}, \ldots\) are defined as follows:
    $$
    a_{0}=1, \quad a_{1}=3, \quad \text { and } a_{n+1}=a_{n}+a_{n-1} \text { for all } n \geq 1
    $$
    Find all integers \(n \geq 1\) for which \(n a_{n+1}+a_{n}\) and \(n a_{n}+a_{n-1}\) share a common factor greater than 1
\end{problem}

\begin{problem}[IrMO 2011 Q6]
    Prove that
    $$
    \frac{2}{3}+\frac{4}{5}+\cdots+\frac{2010}{2011}
    $$
    is not an integer.
\end{problem}

\begin{problem}[IrMO 2011 Q10]
    Find with proof all solutions in nonnegative integers \(a, b, c, d\) of the equation
    $$
    11^{a} 5^{b}-3^{c} 2^{d}=1
    $$
\end{problem}

\begin{problem}[IrMO 2010 Q1]
    Find the least \(k\) for which the number 2010 can be expressed as the sum of the squares of \(k\) integers.
\end{problem}


\begin{problem}[IrMO 2010 Q9]
    Let \(n \geq 3\) be an integer and \(a_{1}, a_{2}, \ldots, a_{n}\) be a finite sequence of positive integers, such that, for \(k=2,3, \ldots, n\)
    $$
    n\left(a_{k}+1\right)-(n-1) a_{k-1}=1
    $$
    Prove that \(a_{n}\) is not divisible by \((n-1)^{2}\)
\end{problem}

\begin{problem}[IrMO 2009 Q3]
    Find all positive integers \(n\) for which \(n^{8}+n+1\) is a prime number.
\end{problem}

\begin{problem}[IrMO 2009 Q7]
    For any positive integer \(n\) define
    $$
    E(n)=n(n+1)(2 n+1)(3 n+1) \cdots(10 n+1)
    $$
    Find the greatest common divisor of \(E(1), E(2), E(3), \ldots, E(2009)\)
\end{problem}

\begin{problem}[IrMO 2008 Q1]
    Let \(p_{1}, p_{2}, p_{3}\) and \(p_{4}\) be four different prime numbers satisfying the equations
$$
\begin{aligned}
2 p_{1}+3 p_{2}+5 p_{3}+7 p_{4} &=162 \\
11 p_{1}+7 p_{2}+5 p_{3}+4 p_{4} &=162
\end{aligned}
$$
Find all possible values of the product \(p_{1} p_{2} p_{3} p_{4}\)
\end{problem}

\begin{problem}[IrMO 2008 Q3]
    Determine, with proof, all integers \(x\) for which \(x(x+1)(x+7)(x+8)\) is a perfect square.
\end{problem}

\begin{problem}[IrMO 2008 Q6]
    Find, with proof, all triples of integers \((a, b, c)\) such that \(a, b\) and \(c\) are the lengths of the sides of a right angled triangle whose area is \(a+b+c\)
\end{problem}

\begin{problem}[IrMO 2007 Q1]
    Find all prime numbers \(p\) and \(q\) such that \(p\) divides \(q+6\) and \(q\) divides \(p+7\)
\end{problem}

\begin{problem}[IrMO 2007 Q5]
    Let \(r\) and \(n\) be nonnegative integers such that \(r \leq n\)
    \begin{enumerate}
    \item Prove that
    $$
    \frac{n+1-2 r}{n+1-r}\left(\begin{array}{l}
    {n} \\
    {r}
    \end{array}\right)
    $$
    is an integer.
    \item Prove that
    $$
    \sum_{r=0}^{\lfloor n / 2\rfloor} \frac{n+1-2 r}{n+1-r}\left(\begin{array}{l}
    {n} \\
    {r}
    \end{array}\right)<2^{n-2}
    $$
    for all \(n \geq 9\)
\end{enumerate}
\end{problem}

\begin{problem}[IrMO 2007 Q9]
    Find the number of zeros in which the decimal expansion of the integer \(2007 !\) ends. Also find its last non-zero digit.
\end{problem}

\begin{problem}[IrMO 2006 Q1]
    Are there integers \(x, y\) and \(z\) which satisfy the equation
    $$
    z^{2}=\left(x^{2}+1\right)\left(y^{2}-1\right)+n
    $$
    when (a) $n=2006$ (b) $n=2007$ ?
\end{problem}

\begin{problem}[IrMO 2009 Q9]
    Let \(n\) be a positive integer. Find the greatest common divisor of the numbers
    $$
    \binom{2n}{1}, \binom{2n}{3}, \binom{2n}{5}, \dots, \binom{2n}{2n - 1}.
    $$
\end{problem}

\begin{problem}[IrMO 2005 Q1]
    Prove that \(2005^{2005}\) is a sum of two perfect squares, but not the sum of two perfect cubes.
\end{problem}

\begin{problem}[IrMO 2005 Q7]
    Using only the digits \(1,2,3,4\) and \(5,\) two players \(A, B\) compose a 2005 -digit number
    \(N\) by selecting one digit at a time as follows: \(A\) selects the first digit, \(B\) the second, \(A\) the third and so on, in that order. The last to play wins if and only if \(N\) is divisible by \(9 .\) Who will win if both players play as well as possible?
\end{problem}

\begin{problem}[IrMO 2005 Q8]
    Suppose that \(x\) is an integer and \(y, z, w\) are odd integers. Show that 17 divides \(x^{y^{z^{w}}}-x^{y^{z}}\)
    
    [Note: Given a sequence of integers \(a_{n}, n=1,2, \ldots,\) the terms \(b_{n}, n=1,2, \ldots,\) of its sequence of "towers" \(a_{1}, a_{2}^{a_{1}}, a_{3}^{a_{2}}, a_{4}^{a_{2}}, \ldots,\) are defined recursively as follows:
    \(\left.b_{1}=a_{1}, b_{n+1}=a_{n+1}^{b_{n}}, n=1,2, \ldots .\right]\)
\end{problem}

\begin{problem}[IrMO 2005 Q9]
    Find the first digit to the left, and the first digit to the right, of the decimal point in the decimal expansion of \((\sqrt{2}+\sqrt{5})^{2000}\)
\end{problem}

\begin{problem}[IrMO 2005 Q10]
    Let \(m, n\) be odd integers such that \(m^{2}-n^{2}+1\) divides \(n^{2}-1 .\) Prove that \(m^{2}-n^{2}+1\)
    is a perfect square.
\end{problem}

\begin{problem}[IrMO 2004 Q1]
    \begin{enumerate}
        \item For which positive integers \(n,\) does \(2 n\) divide the sum of the first \(n\) positive integers?
        \item Determine, with proof, those positive integers \(n\) (if any) which have the property that \(2 n+1\) divides the sum of the first \(n\) positive integers.
    \end{enumerate}
\end{problem}

\begin{problem}[IrMO 2004 Q6]
    Determine all pairs of prime numbers \((p, q),\) with \(2 \leq p, q<100,\) such that \(p+\)
    \(6, p+10, q+4, q+10\) and \(p+q+1\) are all prime numbers.
\end{problem}

\begin{problem}[IrMO 2004 Q8]
    Suppose \(n\) is an integer \(\geq 2 .\) Determine the first digit after the decimal point in the decimal expansion of the number
    $$
    \sqrt[3]{n^{3}+2 n^{2}+n}
    $$
\end{problem}

\begin{problem}[IrMO 2004 Q10]
    Suppose \(p, q\) are distinct primes and \(S\) is a subset of \(\{1,2, \ldots, p-1\} .\) Let \(N(S)\) denote the number of solutions of the equation
    $$
\sum_{i = 1}^{q} x_i \equiv 0 \pmod{p}
    $$
    where \(x_{i} \in S, i=1,2, \ldots, q .\) Prove that \(N(S)\) is a multiple of \(q\)
\end{problem}

\begin{problem}[IrMO 2003 Q1]
    Find all solutions in (not necessarily positive) integers of the equation
    $$
    \left(m^{2}+n\right)\left(m+n^{2}\right)=(m+n)^{3}
    $$
\end{problem}

\begin{problem}[IrMO 2003 Q3]
    For each positive integer \(k,\) let \(a_{k}\) be the greatest integer not exceeding \(\sqrt{k}\) and let \(b_{k}\) be the greatest integer not exceeding \(\sqrt[3]{k} .\) Calculate
    $$
    \sum_{k=1}^{2003}\left(a_{k}-b_{k}\right)
    $$
\end{problem}

\begin{problem}[IrMO 2003 Q8]
    Find all solutions in integers \(x, y\) of the equation
$$
y^{2}+2 y=x^{4}+20 x^{3}+104 x^{2}+40 x+2003
$$
\end{problem}

\begin{problem}[IrMO 2002 Q3]
    Find all triples of positive integers \((p, q, n),\) with \(p\) and \(q\) primes, satisfying
    $$
    p(p+3)+q(q+3)=n(n+3)
    $$
\end{problem}

\begin{problem}[IrMO 2002 Q4]
    Let the sequence \(a_{1}, a_{2}, a_{3}, a_{4}, \ldots\) be defined by
    $$
    a_{1}=1, a_{2}=1, a_{3}=1
    $$
    and
    $$
    a_{n+1} a_{n-2}-a_{n} a_{n-1}=2
    $$
    for all \(n \geq 3 .\) Prove that \(a_{n}\) is a positive integer for all \(n \geq 1\)
\end{problem}

\begin{problem}[IrMO 2002 Q7]
    Suppose \(n\) is a product of four distinct primes \(a, b, c, d\) such that
    \begin{align*}
    a+c&=d
    a(a+b+c+d)&=c(d-b)
    1+b c+d&=b d
    \end{align*}
    Determine \(n\)
\end{problem}

\begin{problem}[IrMO 2002 Q9]
    For each real number \(x,\) define \(\lfloor x\rfloor\) to be the greatest integer less than or equal to
    \(x\)
    Let \(\alpha=2+\sqrt{3} .\) Prove that
    $$
    \alpha^{n}-\left\lfloor\alpha^{n}\right\rfloor= 1-\alpha^{-n}, \text { for } n=0,1,2, \dots
    $$
\end{problem}

\begin{problem}[IrMO 2001 Q1]
    Find, with proof, all solutions of the equation
    $$
    2^{n}=a !+b !+c !
    $$
    in positive integers \(a, b, c\) and \(n .\)
\end{problem}

\begin{problem}[IrMO 2001 Q3]
    Prove that if an odd prime number \(p\) can be expressed in the form \(x^{5}-y^{5},\) for some integers \(x, y,\) then
    $$
    \sqrt{\frac{4 p+1}{5}}=\frac{v^{2}+1}{2}
    $$
    for some odd integer \(v\)
\end{problem}

\begin{problem}[IrMO 2001 Q6]
    Find the least positive integer \(a\) such that 2001 divides \(55^{n}+a 32^{n}\) for some odd integer \(n\)
\end{problem}

\begin{problem}[IrMO 2001 Q9]
    Determine, with proof, all non-negative real numbers \(x\) for which
    $$
    \sqrt[3]{13+\sqrt{x}}+\sqrt[3]{13-\sqrt{x}}
    $$
    is an integer.
\end{problem}

\begin{problem}[IrMO 2000 Q1]
    Let \(S\) be the set of all numbers of the form \(a(n)=n^{2}+n+1,\) where \(n\) is a natural number. Prove that the product \(a(n) a(n+1)\) is in \(S\) for all natural numbers \(n\) Give, with proof, an example of a pair of elements \(s, t \in S\) such that \(s t \notin S .\)
\end{problem}

\begin{problem}[IrMO 2000 Q3]
    Let \(f(x)=5 x^{13}+13 x^{5}+9 a x .\) Find the least positive integer \(a\) such that 65 divides
    \(f(x)\) for every integer \(x\) 
\end{problem}

\begin{problem}[IrMO 2000 Q8]
    For each positive integer \(n\) determine with proof, all positive integers \(m\) such that there exist positive integers \(x_{1}<x_{2}<\cdots<x_{n}\) with
    $$
    \frac{1}{x_{1}}+\frac{2}{x_{2}}+\frac{3}{x_{3}}+\cdots+\frac{n}{x_{n}}=m
    $$
\end{problem}

\begin{problem}[IrMO 2000 Q9]
    Prove that in each set of ten consecutive integers there is one which is coprime with each of the other integers. For example, taking \(114,115,116,117,118,119,120,121,122,123\) the numbers 119 and 121 are each coprime with all the others. [Two integers \(a, b\) are coprime if their greatest common divisor is one. \(]\)
\end{problem}

\begin{problem}[IrMO 1999 Q2]
    Show that there is a positive number in the Fibonacci sequence that is divisible by 1000
\end{problem}

\begin{problem}[IrMO 1999 Q5]
    Three real numbers \(a, b, c\) with \(a<b<c,\) are said to be in arithmetic progression
    if \(c-b=b-a\) Define a sequence \(u_{n}, n=0,1,2,3, \ldots\) as follows: \(u_{0}=0, u_{1}=1\) and, for each \(n \geq\)
    \(1, u_{n+1}\) is the smallest positive integer such that \(u_{n+1}>u_{n}\) and \(\left\{u_{0}, u_{1}, \ldots, u_{n}, u_{n+1}\right\}\) contains no three elements that are in arithmetic progression. Find \(u_{100}\)
\end{problem}

\begin{problem}[IrMO 1999 Q9]
    Find all positive integers \(m\) with the property that the fourth power of the number of (positive) divisors of \(m\) equals \(m\).
\end{problem}

\begin{problem}[IrMO 1998 Q3]
    Show that no integer of the form \(x y x y\) in base \(10,\) where \(x\) and \(y\) are digits, can be the cube of an integer. Find the smallest base \(b>1\) for which there is a perfect cube of the form \(x y x y\) in base \(b\)
\end{problem}

\begin{problem}[IrMO 1998 Q5]
    If \(x\) is a real number such that \(x^{2}-x\) is an integer, and, for some \(n \geq 3, x^{n}-x\) is also an integer, prove that \(x\) is an integer.
\end{problem}

\begin{problem}[IrMO 1998 Q6]
    Find all positive integers \(n\) that have exactly 16 positive integral divisors \(d_{1}, d_{2}\)
    \(\ldots, d_{16}\) such that
    $$
    1=d_{1}<d_{2}<\cdots<d_{16}=n
    $$
    \(d_{6}=18\) and \(d_{9}-d_{8}=17\)
\end{problem}

\begin{problem}[IrMO 1997 Q1]
    Find, with proof, all pairs of integers \((x, y)\) satisfying the equation
    $$
    1+1996 x+1998 y=x y
    $$
\end{problem}

\begin{problem}[IrMO 1997 Q5]
    Let \(S\) be the set of all odd integers greater than one. For each \(x \in S,\) denote by
    \(\delta(x)\) the unique integer satisfying the inequality \(2^{\delta(x)}<x<2^{\delta(x)+1}\)
    For \(a, b \in S,\) define \(a * b=2^{\delta(a)-1}(b-3)+a\)

    [For example, to calculate \(5 * 7,\) note that \(2^{2}<5<2^{3},\) so \(\delta(5)=2,\) and hence \(5 * 7=2^{2-1}(7-3)+5=13 \) Also  \(2^{2}<7<2^{3}\),  so $\delta(7)=2$ and $7 * 5=2^{2-1}(5-3)+7=11$]

    Prove that if \(a, b, c \in S,\) then
    \begin{enumerate}
        \item \(a * b \in S\) and
        \item \((a * b) * c=a *(b * c)\) 
    \end{enumerate}
\end{problem}

\begin{problem}[IrMO 1997 Q6]
    Given a positive integer \(n,\) denote by \(\sigma(n)\) the sum of all positive integers which divide \(n .\) [For example, \(\sigma(3)=1+3=4, \sigma(6)=1+2+3+6=12, \sigma(12)=\)
    \(1+2+3+4+6+12=28]\)
    We say that \(n\) is abundant if \(\sigma(n)>2 n .\) (So, for example, 12 is abundant). Let \(a, b\) be positive integers and suppose that \(a\) is abundant. Prove that \(a b\) is abundant.
\end{problem}

\begin{problem}[IrMO 1996 Q1]
    For each positive integer \(n,\) let \(f(n)\) denote the highest common factor of \(n !+1\) and \((n+1) !\) (where ! denotes factorial). Find, with proof, a formula for \(f(n)\) for each \(n .\)
\end{problem}

\begin{problem}[IrMO 1996 Q2]
    For each positive integer \(n,\) let \(S(n)\) denote the sum of the digits of \(n\) when \(n\) is written in base ten. Prove that, for every positive integer \(n\)
    $$
    S(2 n) \leq 2 S(n) \leq 10 S(2 n)
    $$
    Prove also that there exists a positive integer \(n\) with
    $$
    S(n)=1996 S(3 n)
    $$
\end{problem}

\begin{problem}[IrMO 1996 Q6]
    The Fibonacci sequence \(F_{0}, F_{1}, F_{2}, \ldots\) is defined as follows: \(F_{0}=0, F_{1}=1\) and for all \(n \geq 0\)
    $$
    F_{n+2}=F_{n}+F_{n+1}
    $$

    Prove that
    \begin{enumerate}
        \item The statement ``$F_{n +k} - f_n$ is divisible by 10 for all positive integers $n$'' is true if $k = 60$, but not true for any positive integer $k < 60$.
        \item The statement ``$F_{n + t} - F_n$'' is divisibly by 100 for all positive integers $n$'' is true if $t = 300$, but no true for any positive integer $t < 300$.
    \end{enumerate}
\end{problem}

\begin{problem}[IrMO 1996 Q8]
    Let \(p\) be a prime number, and \(a\) and \(n\) positive integers. Prove that if
    $$
    2^{p}+3^{p}=a^{n}
    $$
    then \(n=1\)
\end{problem}

\begin{problem}[IrMO 1995 Q2]
    Determine, with proof, all those integers \(a\) for which the equation
    $$
    x^{2}+a x y+y^{2}=1
    $$
    Thas infinitely many distinct integer solutions \(x, y .\)
\end{problem}

\begin{problem}[IrMO 1995 Q10]
    For each integer \(n\) such that \(n=p_{1} p_{2} p_{3} p_{4},\) where \(p_{1}, p_{2}, p_{3}, p_{4}\) are distinct primes, let
    $$
    d_{1}=1<d_{2}<d_{3}<\cdots<d_{15}<d_{16}=n
    $$
    be the sixteen positive integers that divide \(n .\) Prove that if \(n<1995,\) then \(d_{9}-d_{8} \neq 22\).
\end{problem}

\begin{problem}[IrMO 1994 Q1]
    Let \(x, y\) be positive integers, with \(y>3,\) and
    $$
    x^{2}+y^{4}=2\left[(x-6)^{2}+(y+1)^{2}\right]
    $$
    Prove that \(x^{2}+y^{4}=1994\)
\end{problem}

\begin{problem}[IrMO 1994 Q6]
    A sequence \(x_{n}\) is defined by the rules: \(x_{1}=2\) and
    $$
    n x_{n}=2(2 n-1) x_{n-1}, \quad n=2,3, \ldots
    $$
    Prove that \(x_{n}\) is an integer for every positive integer \(n\)
\end{problem}

\begin{problem}[IrMO 1993 Q2]
    A natural number \(n\) is called good if it can be written in a unique way simultaneously as the sum \(a_{1}+a_{2}+\ldots+a_{k}\) and as the product \(a_{1} a_{2} \ldots a_{k}\) of some \(k \geq 2\) natural numbers \(a_{1}, a_{2}, \ldots, a_{k} .\) (For example 10 is good because \(10=5+2+1+1+1=5.2 .1 .1 .1\) and these expressions are unique.) Determine, in terms of prime numbers, which natural numbers are good.
\end{problem}

\begin{problem}[IrMO 1992 Q6]
    Let \(n>2\) be an integer and let \(m=\sum k^{3},\) where the sum is taken over all integers
    \(k\) with \(1 \leq k<n\) that are relatively prime to \(n .\) Prove that \(n\) divides \(m .\) (Note that two integers are relatively prime if, and only if, their greatest common divisor equals \(1 .)\)
\end{problem}

\begin{problem}[IrMO 1992 Q7]
    If \(a_{1}\) is a positive integer, form the sequence \(a_{1}, a_{2}, a_{3}, \ldots\) by letting \(a_{2}\) be the product of the digits of \(a_{1},\) etc. If \(a_{k}\) consists of a single digit, for some \(k \geq 1, a_{k}\) is called a digital root of \(a_{1} .\) It is easy to check that every positive integer has a unique digital root. (For example, if \(a_{1}=24378,\) then \(a_{2}=1344, a_{3}=48, a_{4}=32\) \(a_{5}=6,\) and thus 6 is the digital root of \(24378 .\) Prove the digital root of a positive integer \(n\) equals 1 if, and only if, all the digits of \(n\) equal 1
\end{problem}

\begin{problem}[IrMO 1991 Q6]
    The sum of two consecutive squares can be a square: for instance, \(3^{2}+4^{2}=5^{2}\)
    \begin{enumerate}
    \item Prove that the sum of \(m\) consecutive squares cannot be a square for the cases
    $$
    m=3,4,5,6
    $$
    \item Find an example of eleven consecutive squares whose sum is a square.
\end{enumerate}
\end{problem}

\begin{problem}[IrMO 1990 Q2]
    A sequence of primes \(a_{n}\) is defined as follows: \(a_{1}=2,\) and, for all \(n \geq 2, a_{n}\) is the largest prime divisor of \(a_{1} a_{2} \cdots a_{n-1}+1 .\) Prove that \(a_{n} \neq 5\) for all \(n\)
\end{problem}

\begin{problem}[IrMO 1990 Q6]
    Let \(n\) be a natural number, and suppose that the equation
    $$
    x_{1} x_{2}+x_{2} x_{3}+x_{3} x_{4}+x_{4} x_{5}+\cdots+x_{n-1} x_{n}+x_{n} x_{1}=0
    $$
    has a solution with all the \(x_{i}\) 's equal to \(\pm 1 .\) Prove that \(n\) is divisible by \(4 .\)
\end{problem}

\begin{problem}[IRMO 1990 Q7]
    Suppose that \(p_{1}<p_{2}<\ldots<p_{15}\) are prime numbers in arithmetic progression, with common difference \(d .\) Prove that \(d\) is divisible by \(2,3,5,7,11\) and 13
\end{problem}

\begin{problem}[IrMO 1989 Q2]
    A \(3 \times 3\) magic square, with magic number \(m,\) is a \(3 \times 3\) matrix such that the entries on each row, each column and each diagonal sum to \(m .\) Show that if the square has positive integer entries, then \(m\) is divisible by \(3,\) and each entry of the square is at most \(2 n-1,\) where \(m=3 n .\) 
    
    An example of a magic square with \(m=6\) is
$$
\begin{pmatrix}
    {2} & {1} & {3} \\
{3} & {2} & {1} \\
{1} & {3} & {2}
\end{pmatrix}
$$
\end{problem}

\begin{problem}[IrMO 1989 Q4]
    Note that \(12^{2}=144\) end in two 4 's and \(38^{2}=1444\) end in three \(4^{\prime}\) 's. Determine the length of the longest string of equal nonzero digits in which the square of an integer can end.
\end{problem}

\begin{problem}[IrMO 1989 Q5]
    Let \(x=a_{1} a_{2} \ldots a_{n}\) be an \(n\) -digit number, where \(a_{1}, a_{2}, \ldots, a_{n}\left(a_{1} \neq 0\right)\) are the digits. The \(n\) numbers
    $$
    \begin{aligned}
    x_{1}=x=a_{1} a_{2} \ldots a_{n}, & x_{2}=a_{n} a_{1} \ldots a_{n-1}, \quad x_{3}=a_{n-1} a_{n} a_{1} \ldots a_{n-2} \\
    x_{4}=a_{n-2} a_{n-1} a_{n} a_{1} \ldots a_{n-3}, & \ldots, x_{n}=a_{2} a_{3} \ldots a_{n} a_{1}
    \end{aligned}
    $$
    are said to be obtained from \(x\) by the cyclic permutation of digits. [For example, if \(n=5\) and \(x=37001,\) then the numbers are \(x_{1}=37001, x_{2}=13700, x_{3}=01370(=\) \(1370), x_{4}=00137(=137), x_{5}=70013 .\)
    Find, with proof, (i) the smallest natural number \(n\) for which there exists an \(n-\) digit number \(x\) such that the \(n\) numbers obtained from \(x\) by the cyclic permutation
    of digits are all divisible by \(1989 ;\) and (ii) the smallest natural number \(x\) with this property.
\end{problem}

\begin{problem}[IrMO 1989 Q9]
    Let \(a\) be a positive real number, and let
    $$
    b=\sqrt[3]{a+\sqrt{a^{2}+1}}+\sqrt[3]{a-\sqrt{a^{2}+1}}
    $$
    Prove that \(b\) is a positive integer if, and only if, \(a\) is a positive integer of the form \(\frac{1}{2} n\left(n^{2}+3\right),\) for some positive integer \(n\)
\end{problem}

\begin{problem}[IrMO 1988 Q8]
    Let \(x_{1}, x_{2}, x_{3}, \ldots\) be a sequence of nonzero real numbers satisfying
    $$
    x_{n}=\frac{x_{n-2} x_{n-1}}{2 x_{n-2}-x_{n-1}}, \quad n=3,4,5, \ldots
    $$
    Establish necessary and sufficient conditions on \(x_{1}, x_{2}\) for \(x_{n}\) to be an integer for infinitely many values of \(n\)
\end{problem}

\begin{problem}[1988 Q9]
    The year 1978 was "peculiar" in that the sum of the numbers formed with the first two digits and the last two digits is equal to the number formed with the middle two digits, i.e., \(19+78=97 .\) What was the last previous peculiar year, and when will the next one occur?
\end{problem}

\begin{problem}[1988 Q14]
    Let \(x_{1}, \ldots, x_{n}\) be \(n\) integers, and let \(p\) be a positive integer, with \(p<n .\) Put
    $$
    \begin{aligned}
    S_{1} &=x_{1}+x_{2}+\ldots+x_{p} \\
    T_{1} &=x_{p+1}+x_{p+2}+\ldots+x_{n} \\
    S_{2} &=x_{2}+x_{3}+\ldots+x_{p+1} \\
    T_{2} &=x_{p+2}+x_{p+3}+\ldots+x_{n}+x_{1} \\
    \vdots \\
    S_{n} &=x_{n}+x_{1}+x_{2}+\ldots+x_{p-1} \\
    T_{n} &=x_{p}+x_{p+1}+\ldots+x_{n-1}
    \end{aligned}
    $$
    For \(a=0,1,2,3,\) and \(b=0,1,2,3,\) let \(m(a, b)\) be the number of numbers \(i, 1 \leq\) \(i \leq n,\) such that \(S_{i}\) leaves remainder \(a\) on division by 4 and \(T_{i}\) leaves remainder \(b\) on division by 4. Show that \(m(1,3)\) and \(m(3,1)\) leave the same remainder when divided by 4 if, and only if, \(m(2,2)\) is even.
\end{problem}

\end{problems}
\end{document}
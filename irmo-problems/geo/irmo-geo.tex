\documentclass{pset}

\usepackage{xcolor}
\usepackage{hyperref}

\definecolor{hintpink}{RGB}{219, 48, 122}

\course{IrMO Geometry}
% \due{Due on \today}
\name{Adam Kelly}

% \blurb{\textbf{Remark.} This is a collection of all functional equation problems that have appeared in the Irish Mathematical Olympiad.
% The questions are ordered chronologically. All problems are due to their respective creators.}

 \blurb{\textbf{Remark.} This is a collection of all geometry problems that have appeared in the Irish Mathematical Olympiad and the Irish EGMO selection test.
 The questions are ordered chronologically. All problems are due to their respective creators.}


\newenvironment{hint}% environment name
{% begin code
\color{hintpink}
\noindent
  \begin{itshape}\textbf{Hint:}%
}%
{\end{itshape}}% end code

\begin{document}


\section*{EGMO Selection Test Problems}

\begin{problems}

\begin{problem}[EGMO TST 2020]
    \(A, B\) and \(C\) are three points on a circle \(\mathcal{C} .\) Let \(B D\) be the bisector of angle \(\angle A B C\) and \(D E \| A B,\) where \(D,\)
    \(E\) lie also on \(\mathcal{C} .\) Prove that \(|B C|=|D E|\)
\end{problem}

\begin{problem}[EGMO TST 2020]
    Let \(A B C D\) be a convex quadrilateral. Let \(M, N, O, P\) be the midpoints of \(A B, B C, C D\) and \(D A\) respectively.
    \begin{enumerate}
        \item Show that \(M N O P\) is a parallelogram.
        \item Show that the area \([M N O P]\) is \(1 / 2\) of the area \([A B C D]\)
    \end{enumerate}
\end{problem}

\begin{problem}[EGMO TST 2020]
    Let \(A B C\) be a triangle. Let \(\mathcal{C}_{1}\) be the circle which passes through \(A\) and \(B\) and which is tangent to \(B C\). Let \(\mathcal{C}_{2}\) be the circle which passes through \(A\) and \(C\) and which is tangent to \(B C .\) Let \(T\) be the point of intersection (other than \(A)\) of \(\mathcal{C}_{1}\) and \(\mathcal{C}_{2} .\) Show that if \(\angle B A T=\angle C A T\) then \(|B T|=|C T| .\)
\end{problem}
    
\begin{problem}[EGMO TST 2019]
    Let \(A B C D\) be a convex quadrilateral. Suppose that \(A B=C D .\) Prove that
    $
    B C \cdot(\sin \angle B-\sin \angle C)=A D \cdot(\sin \angle A-\sin \angle D)
    $
\end{problem}

\begin{problem}[EGMO TST 2019]
    Let \(A B C\) be a right-angled triangle with hypotenuse \(A B .\) Let \(D\) lie on the segment \(B C\) with \(B D=2 \cdot D C\). Let \(M\) be the midpoint of the hypotenuse \(A B .\) Determine the ratio \(A D / D M .\)
\end{problem}

\begin{problem}[EGMO TST 2019]
    Let \(A B C D\) be a square. Let \(P\) be any point on the circumcircle of \(A B C D\) lying on the arc joining \(A\) to \(B\) (and distinct from \(A\) and \(B\) ). Let \(M\) be the point of intersection of \(D P\) with the diagonal \(A C .\) Let \(N\) be the point of intersection of \(C P\) with the side \(A B .\) Show that \(M N\) is parallel to \(B D .\)
\end{problem}

\begin{problem}[EGMO TST 2018]
    In triangle \(A B C, P\) is a point on \(A B, Q\) is a point on \(A C\) and \(X\) is the point of intersection of the line segments \(P C\) and \(Q B .\) The quadrilateral \(A P X Q\) has area \(4 .\) The triangles \(Q X C\) and \(P X B\) have area 5 and 1 respectively. What
    is the area of the triangle \(A B C\)?
\end{problem}

\begin{problem}[EGMO TST 2018]
    Let \(A, B, C, D\) be four distinct points on a circle (in this order). Let \(P\) be the intersection of \(A D\) and
    \(B C,\) and let \(Q\) be the intersection of \(A B\) and \(C D .\) Prove that the angle bisectors of \(\angle D P C\) and \(A Q D\) are perpendicular.
\end{problem}

\begin{problem}[EGMO TST 2017]
    The diagonals of the convex quadrilateral \(A B C D\) of area 1 intersect at \(O .\) If \(\frac{B O}{D O}=\frac{1}{2}\) and \(\frac{A O}{C O}=\frac{3}{4},\) find the area of triangles \(A O B, B O C, C O D\) and \(D O A\).
\end{problem}

\begin{problem}[EGMO TST 2017]
    \(A B C\) is an acute triangle. The bisector \(A L,\) the altitude \(B H\) and the median \(C M\) are such that \(\angle C A L=\angle A B H=\) \(\angle A C M .\) Find the angles of triangle \(A B C\).
\end{problem}

\begin{problem}[EGMO TST 2016]
    \(A B C\) is an acute triangle and \(D\) is a point on the segment \(B C .\) Two circles \(\mathcal{C}_{1}\) and \(\mathcal{C}_{2}\) passing through \(B, D\) and \(C, D\) respectively intersect for the second time at \(P,\) where \(P\) lies inside of triangle \(A B C .\) Denote by \(R\) the intersection of \(\mathcal{C}_{1}\) and \(A B\) and by \(Q\) the intersection of \(\mathcal{C}_{2}\) and \(A C\). 
    
    Prove that \(P\) lies on the circumcircle of triangle \(Q A R .\)
\end{problem}

\begin{problem}[EGMO TST 2016]
    On sides \(A B, B C\) and \(C A\) of triangle \(A B C\) we consider the points \(M, N\) and \(P\) respectively such that
    $$
    \frac{A M}{M B}=\frac{B N}{N C}=\frac{C P}{P A}=\frac{1}{2}
    $$
    Prove that:
    \begin{enumerate}
        \item $[A M N]=\frac{1}{9}$
        \item $[M N P]=\frac{1}{3}[A B C]$
    \end{enumerate}
\end{problem}

\begin{problem}[EGMO TST 2015]
    In triangle \(A B C\) we denote by \(A^{\prime}, B^{\prime}, C^{\prime}\) the midpoints of sides \(B C, C A\) and \(A B\) respectively. We extend \(A A^{\prime}\) beyond \(A^{\prime}\) with \(A^{\prime} M=A A^{\prime} .\) We extend \(B B^{\prime}\) beyond \(B^{\prime}\) with \(B^{\prime} N=B B^{\prime}\) and extend \(C C^{\prime}\) beyond \(C^{\prime}\) with \(C^{\prime} P=C C^{\prime} .\) Denote by \(G_{1}, G_{2}\) and \(G_{3}\) the centroids of triangles \(M B C, N A C\) and \(P A B\). Prove that triangles \(A B C\) and \(G_{1} G_{2} G_{3}\) have the same area.
\end{problem}

\begin{problem}[EGMO TST 2015]
    A triangle has angles of \(36^{\circ}, 72^{\circ}\) and \(72^{\circ} .\) Prove that it has at least one side whose length is not an integer.
\end{problem}

\begin{problem}[EGMO TST 2014]
    A convex quadrilateral \(A B C D\) is given. On the extended diagonal \(B D\) we consider two points \(D\) and \(E\) such that \(B\) lies between \(D\) and \(E, D\) lies between \(B\) and \(F\) and \(B E=B D=D F .\) Prove that the area of the quadrilateral \(A E C F\) is three times bigger than the area of \(A B C D .\)
\end{problem}

\begin{problem}[EGMO TST 2014]
    The length of each side of a triangle is an integer and is a divisor of the perimeter of the triangle. Prove that the triangle is equilateral.
\end{problem}


\begin{problem}[EGMO TST 2014]
    Let \(G\) be the centroid of triangle \(A B C .\) Denote by \(G_{1}, G_{2}\) and \(G_{3}\) the centroids of triangles \(A B G\) \(B C G\) and \(C A G .\) Prove that $\left[G_{1} G_{2} G_{3}\right]=\frac{1}{9}[A B C]$.
\end{problem}

\begin{problem}[EGMO TST 2013]
    Let \(A B C\) be an isosceles triangle with \(|A B|=|A C| .\) The bisector of the angle \(A B C\) meets the side \(A C\) at the point \(D .\)
    
    Prove that if the triangle \(B C D\) is isosceles, the triangle \(A B D\) must also be isosceles.
\end{problem}

\begin{problem}[EGMO TST 2012]
    Three circles of radii \(1,2,3\) and centres at \(A, B, C\) are mutually tangent. Find the area of the triangle ABC.
\end{problem}

\end{problems}

\clearpage

\section*{IrMO Problems}

\begin{problems}
\begin{problem}[IrMO 2020 Q10]
    Show that there exists a hexagon $A B C D E F$ in the plane such that the distance between every pair of vertices is an integer.
\end{problem}

\begin{problem}[IrMO 2020 Q9]
    A trapezium $A B C D,$ in which $A B$ is parallel to $D C,$ is inscribed in $a$ circle of radius $R$ and centre $O .$ The non-parallel sides $D A$ and $C B$ are extended to meet at $P$ while diagonals $A C$ and $B D$ intersect at $E .$ Prove that $|O E| \cdot|O P|=R^{2}$.
\end{problem}

\begin{problem}[IrMO 2020 Q3]
    Circles $\Omega_{1}$, centre $Q,$ and $\Omega_{2},$ centre $R,$ touch externally at $B .$ A third circle, $\Omega_{3},$ which contains $\Omega_{1}$ and $\Omega_{2},$ touches $\Omega_{1}$ and $\Omega_{2}$ at $A$ and $C,$ respectively. Point $C$ is joined to $B$ and the line $B C$ is extended to meet $\Omega_{3}$ at $D .$

    Prove that $Q R$ and $A D$ intersect on the circumference of $\Omega_{1}$.
\end{problem}

\begin{problem}[IrMO 2019 Q3]
    A quadrilateral \(A B C D\) is such that the sides \(A B\) and \(D C\) are parallel, and \(|B C|=\) \(|A B|+|C D| .\) Prove that the angle bisectors of the angles \(\angle A B C\) and \(\angle B C D\) intersect at right angles on the side AD.
\end{problem}


\begin{problem}[IrMO 2019 Q5]
    Let \(M\) be a point on the side \(B C\) of triangle \(A B C\) and let \(P\) and \(Q\) denote the circumcentres of triangles \(A B M\) and \(A C M\) respectively. Let \(L\) denote the point of intersection of the extended lines \(B P\) and \(C Q\) and let \(K\) denote the reflection of \(L\) through the line \(P Q\)
    
    Prove that \(M, P, Q\) and \(K\) all lie on the same circle.
\end{problem}

\begin{problem}[IrMO 2019 Q8]
    Consider a point \(G\) in the interior of a parallelogram \(A B C D .\) A circle \(\Gamma\) through \(A\) and \(G\) intersects the sides \(A B\) and \(A D\) for the second time at the points \(E\) and \(F\) respectively. The line \(F G\) extended intersects the side \(B C\) at \(H\) and the line \(E G\) extended intersects the side \(C D\) at \(I .\) The circumcircle of triangle \(H G I\) intersects the circle \(\Gamma\) for the second time at \(M \neq G .\) Prove that \(M\) lies on the diagonal \(A C\).
\end{problem}


\begin{problem}[IrMO 2018 Q2]
    The triangle \(A B C\) is right-angled at \(A .\) Its incentre is \(I,\) and \(H\) is the foot of the perpendicular from \(I\) on \(A B .\) The perpendicular from \(H\) on \(B C\) meets \(B C\) at \(E\) and it meets the bisector of \(\angle A B C\) at \(D .\) The perpendicular from \(A\) on \(B C\) meets \(B C\) at \(F .\) Prove that \(\angle E F D=45^{\circ} .\)
\end{problem}

\begin{problem}[IrMO 2018 Q4]
    We say that a rectangle with side lengths \(a\) and \(b\) fits inside a rectangle with side lengths \(c\) and \(d\) if either \((a \leq c \text { and } b \leq d)\) or \((a \leq d \text { and } b \leq c) .\) For instance, a rectangle with side lengths 1 and 5 fits inside a rectangle with side lengths 6 and 2 . Suppose \(S\) is a set of 2019 rectangles, all with integer side lengths between 1 and 2018 inclusive. Show that there are three rectangles \(A, B,\) and \(C\) in \(S\) such that \(A\) fits inside \(B,\) and \(B\) fits inside \(C .\)
\end{problem}

\begin{problem}[IrMO 2018 Q5]
    Points \(A, B\) and \(P\) lie on the circumference of a circle \(\Omega_{1}\) such that \(\angle A P B\) is an obtuse angle. Let \(Q\) be the foot of the perpendicular from \(P\) on \(A B .\) A second circle \(\Omega_{2}\) is drawn with centre \(P\) and radius \(P Q .\) The tangents from \(A\) and \(B\) to
    \(\Omega_{2}\) intersect \(\Omega_{1}\) at \(F\) and \(H\) respectively. Prove that \(F H\) is tangent to \(\Omega_{2}\)
\end{problem}

\begin{problem}[IrMO 2018 Q8]
    Let \(M\) be the midpoint of side \(B C\) of an equilateral triangle \(A B C .\) The point
    \(D\) is on \(C A\) extended such that \(A\) is between \(D\) and \(C .\) The point \(E\) is on \(A B\) extended such that \(B\) is between \(A\) and \(E,\) and \(|M D|=|M E| .\) The point \(F\) is the intersection of \(M D\) and \(A B .\) Prove that \(\angle B F M=\angle B M E\)
\end{problem}


\begin{problem}[IrMO 2017 Q3]
    Four circles are drawn with the sides of the quadrilateral \(A B C D\) as diameters. The two circles passing through \(A\) meet again at \(A^{\prime},\) the two circles through \(B\) at \(B^{\prime}\), the two circles through \(C\) at \(C^{\prime}\) and the two circles through \(D\) at \(D^{\prime} .\) Suppose that the points \(A^{\prime}, B^{\prime}, C^{\prime}\) and \(D^{\prime}\) are distinct. Prove that the quadrilateral \(A^{\prime} B^{\prime} C^{\prime} D^{\prime}\) is similar to the quadrilateral \(A B C D .\)
    
    [Two quadrilaterals are similar if their corresponding angles are equal to each other and their corresponding side lengths are in proportion to each other.]
\end{problem}


\begin{problem}[IrMO 2017 Q8]
    A line segment \(B_{0} B_{n}\) is divided into \(n\) equal parts at points \(B_{1}, B_{2}, \ldots, B_{n 1}\) and
    \(A\) is a point such that \(\angle B_{0} A B_{n}\) is a right angle. Prove that
    $$
    \sum_{k=0}^{n}\left|A B_{k}\right|^{2}=\sum_{k=0}^{n}\left|B_{0} B_{k}\right|^{2}
    $$
\end{problem}

\begin{problem}[IrMO 2016 Q2]
    In triangle \(A B C\) we have \(|A B| \neq|A C| .\) The bisectors of \(\angle A B C\) and \(\angle A C B\) meet
    \(A C\) and \(A B\) at \(E\) and \(F,\) respectively, and intersect at \(I .\) If \(|E I|=|F I|\) find the measure of \(\angle B A C\).
\end{problem}


\begin{problem}[IrMO 2016 Q4]
    Let \(A B C\) be a triangle with \(|A C| \neq|B C| .\) Let \(P\) and \(Q\) be the intersection points of the line \(A B\) with the internal and external angle bisectors at \(C,\) so that \(P\) is between \(A\) and \(B .\) Prove that if \(M\) is any point on the circle with diameter \(P Q\) then \(\angle A M P=\angle B M P\)
\end{problem}

\begin{problem}[IrMO 2016 Q6]
    Triangle \(A B C\) has sides \(a=|B C|>b=|A C| .\) The points \(K\) and \(H\) on the segment \(B C\) satisfy \(|C H|=(a+b) / 3\) and \(|C K|=(a-b) / 3 .\) If \(G\) is the centroid of triangle \(A B C,\) prove that \(\angle K G H=90^{\circ} .\)
\end{problem}

\begin{problem}[IrMO 2016 Q10]
    Let \(A E\) be a diameter of the circumcircle of triangle \(A B C .\) Join \(E\) to the orthocentre, \(H,\) of \(\triangle A B C\) and extend \(E H\) to meet the circle again at \(D .\) Prove that the nine point circle of \(\triangle A B C\) passes through the midpoint of \(H D .\)
    
    [Note. The nine point circle of a triangle is a circle that passes through the midpoints of the sides, the feet of the altitudes and the midpoints of the line segments that join the orthocentre to the vertices.]
\end{problem}


\begin{problem}[IrMO 2015 Q1]
    In the triangle \(A B C,\) the length of the altitude from \(A\) to \(B C\) is equal to \(1 . D\) is the midpoint of \(A C .\) What are the possible lengths of \(B D ?\)
    Two circles \(\mathcal{C}_{1}\) and \(\mathcal{C}_{2},\) with centres at \(D\) and \(E\) respectively, touch at \(B .\) The circle having \(D E\) as diameter intersects the circle \(\mathcal{C}_{1}\) at \(H\) and the circle \(\mathcal{C}_{2}\) at \(K .\) The points \(H\) and \(K\) both lie on the same side of the line \(D E . H K\) extended in both directions meets the circle \(\mathcal{C}_{1}\) at \(L\) and meets the circle \(\mathcal{C}_{2}\) at \(M .\) Prove that
\begin{enumerate}
    \item \(|L H|=|K M|\)
    \item the line through \(B\) perpendicular to \(D E\) bisects \(H K\)    
\end{enumerate}
\end{problem}

\begin{problem}[IrMO 2015 Q8]
    In triangle \(\triangle A B C,\) the angle \(\angle B A C\) is less than \(90^{\circ} .\) The perpendiculars from \(C\) on \(A B\) and from \(B\) on \(A C\) intersect the circumcircle of \(\triangle A B C\) again at \(D\) and \(E\) respectively. If \(|D E|=|B C|,\) find the measure of the angle \(\angle B A C .\)
\end{problem}

\begin{problem}[IrMO 2014 Q3]
    In the triangle \(A B C, D\) is the foot of the altitude from \(A\) to \(B C,\) and \(M\) is the midpoint of the line segment \(B C .\) The three angles \(\angle B A D, \angle D A M\) and \(\angle M A C\) are all equal. Find the angles of the triangle \(A B C\).
\end{problem}

\begin{problem}[IrMO 2014 Q7]
    The square \(A B C D\) is inscribed in a circle with centre \(O .\) Let \(E\) be the midpoint
    of \(A D .\) The line \(C E\) meets the circle again at \(F .\) The lines \(F B\) and \(A D\) meet at
    \(H .\) Prove \(|H D|=2|A H|\).
\end{problem}

\begin{problem}[IrMO 2013 Q3]
    The altitudes of a triangle \(A B C\) are used to form the sides of a second triangle \(A_{1} B_{1} C_{1} .\) The altitudes of \(\triangle A_{1} B_{1} C_{1}\) are then used to form the sides of a third triangle \(A_{2} B_{2} C_{2} .\) Prove that \(\triangle A_{2} B_{2} C_{2}\) is similar to \(\triangle A B C .\)
\end{problem}

\begin{problem}[IrMO 2013 Q5]
    \(A, B\) and \(C\) are points on the circumference of a circle with centre \(O .\) Tangents are drawn to the circumcircles of triangles \(O A B\) and \(O A C\) at \(P\) and \(Q\) respectively, where \(P\) and \(Q\) are diametrically opposite \(O .\) The two tangents intersect at \(K\) The line \(C A\) meets the circumcircle of \(\triangle O A B\) at \(A\) and \(X .\) Prove that \(X\) lies on the line \(K O\)
\end{problem}

\begin{problem}[IrMO 2013 Q6]
    The three distinct points \(B, C, D\) are collinear with \(C\) between \(B\) and \(D .\) Another point \(A\) not on the line \(B D\) is such that \(|A B|=|A C| .\) Prove that \(\angle B A C=\) \(36^{\circ}\) if and only if
    $$
    \frac{1}{|C D|}-\frac{1}{|B D|}=\frac{1}{|C D|+|B D|}
    $$
\end{problem}


\begin{problem}[IrMO 2012 Q2]
    \(A, B, C\) and \(D\) are four points in that order on the circumference of a circle \(K\)
    \(A B\) is perpendicular to \(B C\) and \(B C\) is perpendicular to \(C D . X\) is a point on the circumference of the circle between \(A\) and \(D . A X\) extended meets \(C D\) extended
    at \(E\) and \(D X\) extended meets \(B A\) extended at \(F\)

    Prove that the circumcircle of triangle \(A X F\) is tangent to the circumcircle of triangle \(D X E\) and that the common tangent line passes through the centre of the circle \(K\)
\end{problem}

\begin{problem}[IrMO 2012 Q4]
    There exists an infinite set of triangles with the following properties:
    \begin{enumerate}
    \item the lengths of the sides are integers with no common factors, and
    \item one and only one angle is \(60^{\circ} .\)
    \end{enumerate}
    One such triangle has side lengths \(5,7\) and \(8 .\) Find two more.
\end{problem}


\begin{problem}[IrMO 2012 Q7]
    Consider a triangle \(A B C\) with \(|A B| \neq|A C| .\) The angle bisector of the angle \(C A B\) intersects the circumcircle of \(\triangle A B C\) at two points \(A\) and \(D .\) The circle of centre
    \(D\) and radius \(|D C|\) intersects the line \(A C\) at two points \(C\) and \(B^{\prime} .\) The line \(B B^{\prime}\) intersects the circumcircle of \(\triangle A B C\) at \(B\) and \(E .\) Prove that \(B^{\prime}\) is the orthocentre
    of \(\triangle A E D\)
\end{problem}

\begin{problem}[IrMO 2011 Q2]
    Let \(A B C\) be a triangle whose side lengths are, as usual, denoted by \(a=|B C|\) \(b=|C A|, c=|A B| .\) Denote by \(m_{a}, m_{b}, m_{c},\) respectively, the lengths of the medians which connect \(A, B, C,\) respectively, with the centres of the corresponding opposite sides.
    \begin{enumerate}
        \item Prove that \(2 m_{a}<b+c .\) Deduce that \(m_{a}+m_{b}+m_{c}<a+b+c\)
    \item Give an example of
    \begin{enumerate}
    \item a triangle in which \(m_{a}>\sqrt{b c}\)
    \item a triangle in which \(m a \leq \sqrt{b c}\)
    \end{enumerate} 
    \end{enumerate}
\end{problem}

\begin{problem}[IrMO 2011 Q4]
    The incircle \(\mathcal{C}_{1}\) of triangle \(A B C\) touches the sides \(A B\) and \(A C\) at the points \(D\) and \(E,\) respectively. The incircle \(\mathcal{C}_{2}\) of the triangle \(A D E\) touches the sides \(A B\) and \(A C\) at the points \(P\) and \(Q,\) and intersects the circle \(\mathcal{C}_{1}\) at the points \(M\) and \(N .\) Prove that
    \begin{enumerate}
        \item the centre of the circle \(\mathcal{C}_{2}\) lies on the circle \(\mathcal{C}_{1}\)
        \item the four points \(M, N, P, Q\) in appropriate order form a rectangle if and only if twice the radius of \(\mathcal{C}_{1}\) is three times the radius of \(\mathcal{C}_{2}\)
     
    \end{enumerate}
\end{problem}

\begin{problem}[IrMO 2011 Q8]
    \(A B C D\) is a rectangle. \(E\) is a point on \(A B\) between \(A\) and \(B,\) and \(F\) is a point on
    \(A D\) between \(A\) and \(D .\) The area of the triangle \(E B C\) is \(16,\) the area of the triangle \(E A F\) is 12 and the area of the triangle \(F D C\) is \(30 .\) Find the area of the triangle \(E F C\)
\end{problem}

\begin{problem}[IrMO 2010 Q2]
    Let \(A B C\) be a triangle and let \(P\) denote the midpoint of the side \(B C .\) Suppose that there exist two points \(M\) and \(N\) interior to the sides \(A B\) and \(A C\) respectively, such that
    $$
    |A D|=|D M|=2|D N|
    $$
    where \(D\) is the intersection point of the lines \(M N\) and \(A P .\) Show that \(|A C|=|B C|\)
\end{problem}


\begin{problem}[IrMO 2010 Q8]
    In the triangle \(A B C\) we have \(|A B|=1\) and \(\angle A B C=120^{\circ} .\) The perpendicular line to \(A B\) at \(B\) meets \(A C\) at \(D\) such that \(|D C|=1 .\) Find the length of \(A D .\)
\end{problem}

\begin{problem}[IrMO 2010 Q10]
    Suppose \(a, b, c\) are the side lengths of a triangle \(A B C .\) Show that
    $$
    x=\sqrt{a(b+c-a)}, \quad y=\sqrt{b(c+a-b)}, \quad z=\sqrt{c(a+b-c)}
    $$
    are the side lengths of an acute-angled triangle \(X Y Z,\) with the same area as \(A B C,\) but with a smaller perimeter, unless \(A B C\) is equilateral.
\end{problem}


\begin{problem}[IrMO 2009 Q2]
    Let \(A B C D\) be a square. The line segment \(A B\) is divided internally at \(H\) so that \(|A B| \cdot|B H|=|A H|^{2} .\) Let \(E\) be the mid point of \(A D\) and \(X\) be the midpoint of AH. Let \(Y\) be a point on \(E B\) such that \(X Y\) is perpendicular to \(B E .\) Prove that \(|X Y|=|X H|\)
\end{problem}

\begin{problem}[IrMO 2009 Q10]
    In the triangle \(A B C\) we have \(|A B|<|A C| .\) The bisectors of the angles at \(B\) and \(C\) meet \(A C\) and \(A B\) at \(D\) and \(E\) respectively. \(B D\) and \(C E\) intersect at the incentre
    \(I\) of \(\triangle A B C\)

    Prove that \(\angle B A C=60^{\circ}\) if and only if \(|I E|=|I D|\)
\end{problem}

\begin{problem}[IrMO 2008 Q5]
    A triangle \(A B C\) has an obtuse angle at \(B .\) The perpendicular at \(B\) to \(A B\) meets \(A C\) at \(D,\) and \(|C D|=|A B| .\) Prove that \(|A D|^{2}=|A B| \cdot|B C|\) if and only if \(\angle C B D=30^{\circ}\)
\end{problem}

\begin{problem}[IrMO 2008 Q7]
    Circles \(S\) and \(T\) intersect at \(P\) and \(Q,\) with \(S\) passing through the centre of \(T\). Distinct points \(A\) and \(B\) lie on \(S,\) inside \(T,\) and are equidistant from the centre of
    T. The line \(P A\) meets \(T\) again at \(D .\) Prove that \(|A D|=|P B| .\)
\end{problem}

\begin{problem}[IrMO 2007 Q2]
    Prove that a triangle \(A B C\) is right-angled if and only if
    $$
    \sin ^{2} A+\sin ^{2} B+\sin ^{2} C=2
    $$
\end{problem}

\begin{problem}[IrMO 2007 Q3]
    The point \(P\) is a fixed point on a circle and \(Q\) is a fixed point on a line. The point \(R\) is a variable point on the circle such that \(P, Q\) and \(R\) are not collinear. The circle through \(P, Q\) and \(R\) meets the line again at \(V .\) Show that the line \(V R\) passes through a fixed point.
\end{problem}

\begin{problem}[IrMO 2007 Q8]
    Let \(A B C\) be a triangle the lengths of whose sides \(B C, C A, A B,\) respectively, are denoted by \(a, b, c,\) respectively. Let the internal bisectors of the angles \(\angle B A C, \angle A B C, \angle B C A\) respectively, meet the sides \(B C, C A, A B,\) respectively, at \(D, E, F,\) respectively. Denote the lengths of the line segments \(A D, B E, C F,\) respectively, by \(d, e, f,\) respectively. Prove that
    $$
    def=\frac{4 a b c(a+b+c) \Delta}{(a+b)(b+c)(c+a)}
    $$
    where \(\Delta\) stands for the area of the triangle \(A B C .\)
\end{problem}


\begin{problem}[IrMO 2006 Q2]
    \(P\) and \(Q\) are points on the equal sides \(A B\) and \(A C\) respectively of an isosceles triangle \(A B C\) such that \(A P=C Q .\) Moreover, neither \(P\) nor \(Q\) is a vertex of \(A B C .\) Prove that the circumcircle of the triangle \(A P Q\) passes through the circumcentre of the triangle \(A B C .\)
\end{problem}

\begin{problem}[IrMO 2006 Q3]
    Prove that a square of side 2.1 units can be completely covered by seven squares of side 1 unit.
\end{problem}

\begin{problem}[IrMO 2006 Q7]
    \(A B C\) is a triangle with points \(D, E\) on \(B C,\) with \(D\) nearer \(B ; F, G\) on \(A C,\) with \(F\) nearer \(C ; H, K\) on \(A B,\) with \(H\) nearer \(A .\) Suppose that \(A H=A G=1\) \(B K=B D=2, C E=C F=4, \angle B=60^{\circ}\) and that \(D, E, F, F, G, H\) and \(K\) all lie on a circle. Find the radius of the incircle of the triangle \(A B C .\)
\end{problem}

\begin{problem}[IrMO 2005 Q2]
    Let \(A B C\) be a triangle and let \(D, E\) and \(F,\) respectively, be points on the sides
    \(B C, C A\) and \(A B,\) respectively-none of which coincides with a vertex of the triangle such that \(A D, B E\) and \(C F\) meet at a point \(G .\) Suppose the triangles \(A G F, C G E\) and \(B G D\) have equal area. Prove that \(G\) is the centroid of \(A B C\).
\end{problem}

\begin{problem}[IrMO 2005 Q3]
    Prove that the sum of the lengths of the medians of a triangle is at least three quarters of the sum of the lengths of the sides.
\end{problem}

\begin{problem}[IrMO 2005 Q6]
    Let \(A B C\) be a triangle, and let \(X\) be a point on the side \(A B\) that is not \(A\) or \(B .\) Let \(P\) be the incentre of the triangle \(A C X, Q\) the incentre of the triangle \(B C X\) and \(M\) the midpoint of the segment \(P Q .\) Show that \(|M C|>|M X| .\)
\end{problem}

\begin{problem}[IrMO 2004 Q3]
    \(A B\) is a chord of length 6 of a circle centred at \(O\) and of radius \(5 .\) Let \(P Q R S\) denote the square inscribed in the sector \(O A B\) such that \(P\) is on the radius \(O A\) \(S\) is on the radius \(O B\) and \(Q\) and \(R\) are points on the arc of the circle between \(A\) and \(B .\) Find the area of \(P Q R S\).
\end{problem}

\begin{problem}[IrMO 2004 Q2]
    \(A\) and \(B\) are distinct points on a circle \(T . C\) is a point distinct from \(B\) such that \(|A B|=|A C|,\) and such that \(B C\) is tangent to \(T\) at \(B .\) Suppose that the bisector
    of \(\angle A B C\) meets \(A C\) at a point \(D\) inside \(T .\) Show that \(\angle A B C>72^{\circ} .\)
\end{problem}

\begin{problem}[IrMO 2003 Q2]
    \(P, Q, R\) and \(S\) are (distinct) points on a circle. \(P S\) is a diameter and \(Q R\) is parallel to the diameter \(P S . P R\) and \(Q S\) meet at \(A .\) Let \(O\) be the centre of the circle and let \(B\) be chosen so that the quadrilateral \(P O A B\) is a parallelogram. Prove that \(B Q=B P\)
\end{problem}

\begin{problem}[IrMO 2003 Q6]
    Let \(T\) be a triangle of perimeter \(2,\) and let \(a, b\) and \(c\) be the lengths of the sides of \(T\)
    \begin{enumerate}
        \item Show that
        $$
        a b c+\frac{28}{27} \geq a b+b c+a c
        $$
        \item Show that
        $$
        a b+b c+a c \geq a b c+1
        $$
    \end{enumerate}
    
\end{problem}

\begin{problem}[IrMO 2003 Q7]
    \(A B C D\) is a quadrilateral. \(P\) is at the foot of the perpendicular from \(D\) to \(A B, Q\) is at the foot of the perpendicular from \(D\) to \(B C, R\) is at the foot of the perpendicular from \(B\) to \(A D\) and \(S\) is at the foot of the perpendicular from \(B\) to \(C D .\) Suppose that \(\angle P S R=\angle S P Q .\) Prove that \(P R=S Q\)
\end{problem}


\begin{problem}[IrMO 2002 Q1]
    In a triangle \(A B C, A B=20, A C=21\) and \(B C=29 .\) The points \(D\) and \(E\) lie on the line segment \(B C,\) with \(B D=8\) and \(E C=9 .\) Calculate the angle \(\angle D A E\)
\end{problem}

\begin{problem}[IrMO 2002 Q10]
    Let \(A B C\) be a triangle whose side lengths are all integers, and let \(D\) and \(E\) be the points at which the incircle of \(A B C\) touches \(B C\) and \(A C\) respectively. If \(\left.|| A D\right|^{2}-|B E|^{2} | \leq 2,\) show that \(|A C|=|B C|\)
\end{problem}

\begin{problem}[IrMO 2001 Q2]
    Let \(A B C\) be a triangle with sides \(B C, C A, A B\) of lengths \(a, b, c,\) respectively. Let
    \(D, E\) be the midpoints of the sides \(A C, A B,\) respectively. Prove that \(B D\) is perpendicular to \(C E\) if, and only if,
    $$
    b^{2}+c^{2}=5 a^{2}
    $$
\end{problem}

\begin{problem}[IrMO 2001 Q8]
    Let \(A B C\) be an acute angled triangle, and let \(D\) be the point on the line \(B C\) for which \(A D\) is perpendicular to \(B C .\) Let \(P\) be a point on the line segment \(A D .\) The lines \(B P\) and \(C P\) intersect \(A C\) and \(A B\) at \(E\) and \(F\) respectively. Prove that the line \(A D\) bisects the angle \(E D F\)
\end{problem}

\begin{problem}[IrMO 2000 Q2]
Let $ABCDE$ be a regular pentagon with its sides of length one.
Let $F$ be the midpoint of $AB$ and let $G, H$ be points on the sides $CD$ and $DE$, respectively,
such that $\angle GFD = \angle HFD = 30^{\circ}$.
Prove that the triangle $GFH$ is equilateral. A square is inscribed in the triangle $GFH$ with one side of the square along $GH$. PRove that $FG$ has length
$$
t=\frac{2 \cos 18^{\circ}\left(\cos 36^{\circ}\right)^{2}}{\cos 6^{\circ}}
$$
and that the square has sides of length
$$
\frac{t \sqrt{3}}{2+\sqrt{3}}
$$
\end{problem}

\begin{problem}[IrMO 2000 Q5]
    Consider all parabolas of the form \(y=x^{2}+2 p x+q(p, q\) real) which intersect the \(x-\) and \(y\) -axes in three distinct points. For such a pair \(p, q\) let \(C_{p, q}\) be the circle through the points of intersection of the parabola \(y=x^{2}+2 p x+q\) with the axes. Prove that all the circles \(C_{p, q}\) have a point in common.
\end{problem}

\begin{problem}[IrMO 2000 Q7]
    Let \(A B C D\) be a cyclic quadrilateral and \(R\) the radius of the circumcircle. Let \(a, b, c, d\) be the lengths of the sides of \(A B C D\) and \(Q\) its area. Prove that
    $$
    R^{2}=\frac{(a b+c d)(a c+b d)(a d+b c)}{16 Q^{2}}
    $$
    Deduce that 
    $$
    {\qquad R \geq \frac{(a b c d)^{3 / 4}}{Q \sqrt{2}}}
    $$
    with equality if and only if \(A B C D\) is a square.
\end{problem}

\begin{problem}[IrMO 1999 Q3]
    Let \(D, E\) and \(F,\) respectively, be points on the sides \(B C, C A\) and \(A B,\) respectively, of a triangle \(A B C\) so that \(A D\) is perpendicular to \(B C, B E\) is the angle-bisector of \(\angle B\) and \(F\) is the mid-point of \(A B .\) Prove that \(A D, B E\) and \(C F\) are concurrent if and only if,
    $$
    a^{2}(a-c)=\left(b^{2}-c^{2}\right)(a+c)
    $$
    where \(a, b\) and \(c\) are the lengths of the sides \(B C, C A\) and \(A B,\) respectively, of the triangle \(A B C\)
\end{problem}

\begin{problem}[IrMO 1999 Q10]
    \(A B C D E F\) is a convex (not necessarily regular) hexagon with \(A B=B C, C D=\) \(D E, E F=F A\) and
    $$
    \angle A B C+\angle C D E+\angle E F A=360^{\circ}
    $$
    Prove that the perpendiculars from \(A, C\) and \(E\) to \(F B, B D\) and \(D F,\) respectively, are concurrent.
\end{problem}

\begin{problem}[IrMO 1998 Q2]
    \(P\) is a point inside an equilateral triangle such that the distances from \(P\) to the three vertices are \(3,4\) and \(5,\) respectively. Find the area of the triangle.
\end{problem}

\begin{problem}[IrMO 1998 Q4]
    Show that a disc of radius 2 can be covered by seven (possibly overlapping) discs
    of radius 1
\end{problem}

\begin{problem}[IrMO 1998 Q10]
    A triangle \(A B C\) has positive integer sides, \(\angle A=2 \angle B\) and \(\angle C>90^{\circ} .\) Find the minimum length of its perimeter.
\end{problem}

\begin{problem}[IrMO 1997 Q2]
    Let \(A B C\) be an equilateral triangle. For a point \(M\) inside \(A B C,\) let \(D, E, F\) be the feet of the perpendiculars from \(M\) onto \(B C, C A, A B,\) respectively. Find the locus of all such points \(M\) for which \(\angle F D E\) is a right-angle.
\end{problem}

\begin{problem}[IrMO 1997 Q7]
    \(A B C D\) is a quadrilateral which is circumscribed about a circle \(\Gamma\) (i.e., each side of the quadrilateral is tangent to \(\Gamma .\) ) If \(\angle A=\angle B=120^{\circ}, \angle D=90^{\circ}\) and \(B C\) has length \(1,\) find, with proof, the length of \(A D .\)
\end{problem}

\begin{problem}[IrMO 1996 Q4]
    Let \(F\) be the mid-point of the side \(B C\) of a triangle \(A B C .\) Isosceles right-angled triangles \(A B D\) and \(A C E\) are constructed externally on the sides \(A B\) and \(A C\) with right-angles at \(D\) and \(E\) respectively. Prove that \(D E F\) is an isosceles right-angled triangle.
\end{problem}

\begin{problem}[IrMO 1996 Q5]
    Show, with proof, how to dissect a square into at most five pieces in such a way that the pieces can be re-assembled to form three squares no two of which are the same size.
\end{problem}

\begin{problem}[IrMO 1996 Q10]
    Let \(A B C\) be an acute-angled triangle and let \(D, E, F\) be the feet of the perpendiculars from \(A, B, C\) onto the sides \(B C, C A, A B,\) respectively. Let \(P, Q, R\) be the feet of the perpendiculars from \(A, B, C\) onto the lines \(E F, F D, D E,\) respectively. Prove that the lines \(A P, B Q, C R\) (extended) are concurrent.
\end{problem}

\begin{problem}[IrMO 1995 Q3]
    Let \(A, X, D\) be points on a line, with \(X\) between \(A\) and \(D .\) Let \(B\) be a point in the plane such that \(\angle A B X\) is greater than \(120^{\circ},\) and let \(C\) be a point on the line between \(B\) and \(X .\) Prove the inequality
    $$
    2|A D| \geq \sqrt{3}(|A B|+|B C|+|C D|)
    $$
\end{problem}

\begin{problem}[IrMO 1995 Q8]
    Let \(S\) be the square consisting of all points \((x, y)\) in the plane with \(0 \leq x, y \leq 1\) For each real number \(t\) with \(0<t<1,\) let \(C_{t}\) denote the set of all points \((x, y) \in S\) such that \((x, y)\) is on or above the line joining \((t, 0)\) to \((0,1-t)\) Prove that the points common to all \(C_{t}\) are those points in \(S\) that are on or above the curve \(\sqrt{x}+\sqrt{y}=1\)
\end{problem}

\begin{problem}[IrMO 1995 Q9]
    We are given three points \(P, Q, R\) in the plane. It is known that there is a triangle
    \(A B C\) such that \(P\) is the mid-point of the side \(B C, Q\) is the point on the side \(C A\) with \(|C Q| /|Q A|=2,\) and \(R\) is the point on the side \(A B\) with \(|A R| /|R B|=2\) Determine, with proof, how the triangle \(A B C\) may be constructed from \(P, Q, R .\)
\end{problem}


\begin{problem}[IrMO 1994 Q2]
    Let \(A, B, C\) be three collinear points, with \(B\) between \(A\) and \(C .\) Equilateral triangles
    \(A B D, B C E, C A F\) are constructed with \(D, E\) on one side of the line \(A C\) and \(F\) on the opposite side. Prove that the centroids of the triangles are the vertices of an equilateral triangle. Prove that the centroid of this triangle lies on the line \(A C .\)
\end{problem}

\begin{problem}[IrMO 1993 Q3]
    The line \(l\) is tangent to the circle \(S\) at the point \(A ; B\) and \(C\) are points on \(l\) on opposite sides of \(A\) and the other tangents from \(B, C\) to \(S\) intersect at a point \(P .\)
    If \(B, C\) vary along \(l\) in such a way that the product \(|A B| \cdot|A C|\) is constant, find the locus of \(P\)
\end{problem}

\begin{problem}[IrMO 1993 Q6]
    Given five points \(P_{1}, P_{2}, P_{3}, P_{4}, P_{5}\) in the plane having integer coordinates, prove that there is at least one pair \(\left(P_{i}, P_{j}\right),\) with \(i \neq j,\) such that the line \(P_{i} P_{j}\) contains
    a point \(Q\) having integer coordinates and lying strictly between \(P_{i}\) and \(P_{j}\).
\end{problem}

\begin{problem}[IrMO 1992 Q1]
    Describe in geometric terms the set of points \((x, y)\) in the plane such that \(x\) and \(y\) satisfy the condition \(t^{2}+y t+x \geq 0\) for all \(t\) with \(-1 \leq t \leq 1\)
\end{problem}

\begin{problem}[IrMO 1992 Q4]
    In a triangle \(A B C,\) the points \(A^{\prime}, B^{\prime}\) and \(C^{\prime}\) on the sides opposite \(A, B\) and \(C\) respectively, are such that the lines \(A A^{\prime}, B B^{\prime}\) and \(C C^{\prime}\) are concurrent. Prove that the diameter of the circumscribed circle of the triangle \(A B C\) equals the product \(\left|A B^{\prime}\right| \cdot\left|B C^{\prime}\right| \cdot\left|C A^{\prime}\right|\) divided by the area of the triangle \(A^{\prime} B^{\prime} C^{\prime}\)
\end{problem}

\begin{problem}[IrMO 1992 Q5]
    Let \(A B C\) be a triangle such that the coordinates of the points \(A\) and \(B\) are rational numbers. Prove that the coordinates of \(C\) are rational if, and only if, \(\tan A, \tan B\) and tan \(C,\) when defined, are all rational numbers.
\end{problem}

\begin{problem}[IrMO 1992 Q9]
    A convex pentagon has the property that each of its diagonals cuts off a triangle of unit area. Find the area of the pentagon.
\end{problem}

\begin{problem}[IrMO 1991 Q1]
    Three points \(X, Y\) and \(Z\) are given that are, respectively, the circumcentre of a triangle \(A B C,\) the mid-point of \(B C,\) and the foot of the altitude from \(B\) on \(A C\). Show how to reconstruct the triangle \(A B C\).
\end{problem}

\begin{problem}[IrMO 1991 Q8]
    Let \(A B C\) be a triangle and \(L\) the line through \(C\) parallel to the side \(A B .\) Let the internal bisector of the angle at \(A\) meet the side \(B C\) at \(D\) and the line \(L\) at \(E,\) and let the internal bisector of the angle at \(B\) meet the side \(A C\) at \(F\) and the line \(L\) at
    \(G .\) If \(|G F|=|D E|,\) prove that \(|A C|=|B C|\)
\end{problem}


\begin{problem}[IrMO 1990 Q5]
    Let \(A B C\) be a right-angled triangle with right-angle at \(A .\) Let \(X\) be the foot of the perpendicular from \(A\) to \(B C,\) and \(Y\) the mid-point of \(X C .\) Let \(A B\) be extended to
    \(D\) so that \(|A B|=|B D| .\) Prove that \(D X\) is perpendicular to \(A Y\)
\end{problem}

\begin{problem}[IrMO 1989 Q6]
    Suppose \(L\) is a fixed line, and \(A\) a fixed point not on \(L .\) Let \(k\) be a fixed nonzero real number. For \(P\) a point on \(L,\) let \(Q\) be a point on the line \(A P\) with \(|A P| .|A Q|=k^{2}\) Determine the locus of \(Q\) as \(P\) varies along the line \(L\)
\end{problem}

\begin{problem}[IrMO 1988 Q1]
    A pyramid with a square base, and all its edges of length \(2,\) is joined to a regular tetrahedron, whose edges are also of length \(2,\) by gluing together two of the triangular faces. Find the sum of the lengths of the edges of the resulting solid.
\end{problem}

\begin{problem}[IrMO 1988 Q2]
    \(A, B, C, D\) are the vertices of a square, and \(P\) is a point on the arc \(C D\) of its circumcircle. Prove that
$$
|P A|^{2}-|P B|^{2}=|P B| \cdot|P D|-|P A| \cdot|P C|
$$
\end{problem}

\begin{problem}[IrMO 1988 Q3]
    \(A B C\) is a triangle inscribed in a circle, and \(E\) is the mid-point of the arc subtended by \(B C\) on the side remote from \(A .\) If through \(E\) a diameter \(E D\) is drawn, show that the measure of the angle \(D E A\) is half the magnitude of the difference of the measures of the angles at \(B\) and \(C\).
\end{problem}

\begin{problem}[IrMO 1988 Q4]
    A mathematical moron is given the values \(b, c, A\) for a triangle \(A B C\) and is required to find \(a .\) He does this by using the cosine rule
    $$
    a^{2}=b^{2}+c^{2}-2 b c \cos A
    $$
    and misapplying the low of the logarithm to this to get
    $$
    \log a^{2}=\log b^{2}+\log c^{2}-\log (2 b c \cos A)
    $$
    He proceeds to evaluate the right-hand side correctly, takes the anti-logarithms and gets the correct answer. What can be said about the triangle \(A B C ?\)
\end{problem}


\end{problems}

\end{document}